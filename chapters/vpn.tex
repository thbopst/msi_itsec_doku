\chapter{Client-to-LAN VPN}

\section{Aufgabenstellung}

Mitarbeiter von Firma A, sollen die Möglichkeit haben von überall arbeiten zu können und von dort auch Zugriff auf das interne Firmen Netzwerk zu erhalten.
Vorrausgesetzt wird hierbei selbstverständlich der Zugang zum Internet.

Realisiert werden soll dies mittels einer Client-to-LAN VPN-Verbindung die technisch mittels der freien Software \emph{OpenVPN} implementiert werden soll.

\section{Implementierung}

Um eine Verbindung zwischen einem externen Mitarbeiter bzw. dem externen Gerät und dem internen Netzwerk von Firma A herstellen zu können muss einerseits ein \textbf{Server} den VPN Dienst anbieten und andererseits muss der \textbf{Client} so konfiguriert sein, dass er sich mit diesem Dienst verbinden kann.
Als Server dient die innere Firewall \emph{fw2.firma-a.f223}, denn nur auf dieser besteht die Möglichkeit den VPN-Server so einzurichten, damit die sich verbindenden Clients zugriff auf das Intranet erhalten.
Beim Client handelt es sich um den externen Laptop \emph{lap01.internet.f223}.

Im Folgenden wird die Installation und Konfiguration der OpenVPN Software auf Client und Server erläutert.


\subsection{Server}

Auf dem der internen Firewall von Firma a müssen zunänchst die benötigten Software Packete installiert werden. Hierbei handelt es sich um das Packet \emph{openvpn}, welches die OpenVPN Software enthält. Zusätzlich wird noch das Packet \emph{bridge-utils} benötigt, welches zur Konfiguration der Bridge zwischen des von OpenVPN bereitgestellten Neztwerkinterfaces \emph{tap0} und dem normalen Netzwerkadapter \emph{eth0}, mit dem das Intranet verbunden ist, zu verbinden.

Die Installation der Packete erfolgt mittels der folgenden Befehle.

\begin{lstlisting}
apt-get install openvpn
apt-get install bridge-utils
\end{lstlisting}

% Screenshot der ifconfig wenn VPN gestartet ist und wenn nicht

\subsection{Client}
